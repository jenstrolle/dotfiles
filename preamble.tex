\documentclass[a4paper,oneside,article]{memoir}


%%%%%%%%%%%%%%%%%%%%%%
%  Prettier writing  %
%%%%%%%%%%%%%%%%%%%%%%

\usepackage[utf8]{inputenc}      	% Correct handling of æ, ø, å
\usepackage[T1]{fontenc}  		% Correct handling of æ, ø, å
\usepackage{microtype}  		% Better hyphenation
\usepackage[english]{babel}     % Better language support
% \renewcommand{\danishhyphenmins}{22}  % Better danish hyphenation

\usepackage{enumitem} 			% Support for alphabetical enumeration

% \usepackage[notextcomp]{kpfonts} 	% Change font to M&I font
% \usepackage{textcomp} 		% More symbols

\usepackage{parskip}   			% Don't indent paragraphs, leave space between
\usepackage{emptypage} 			% Hide page number when page is empty
\usepackage{xcolor}    			% More colors
\usepackage{subcaption}			% Makes subcaptions in figures possible
\usepackage{tabularx}


%%%%%%%%%%%%%
%  Figures  %
%%%%%%%%%%%%%

% Figure support as explained in Castieldev blogpost.
\usepackage{import}
\usepackage{xifthen}
\usepackage{pdfpages}
\usepackage{transparent}
\newcommand{\incfig}[1]{%
    \def\svgwidth{\columnwidth}
    \import{./figures/}{#1.pdf_tex}
}


%%%%%%%%%%%%%%%%
%  Math stuff  %
%%%%%%%%%%%%%%%%

\usepackage{amsmath}  			% Access to invaluable math stuff
\usepackage{amssymb}  			% Access to more math symbols
\usepackage{amsfonts} 			% More math fonts
\usepackage{amsthm}   			% Access to theoremstyles
\usepackage{xfrac}    			% More options for fractions
\usepackage{mathtools}			% Tools for amsmath, see https://bit.ly/3rE9g0E
\usepackage{cancel}   			% For cancelling things in fractions
\usepackage{bm}       			% Bold math

%%%%%%%%%%%%%%%%
%  Code stuff  %
%%%%%%%%%%%%%%%%

\usepackage{listings}  			% Write code inline
\usepackage{tikz}  			% For drawing figures in TeX
\usepackage{algpseudocode}  		% Prettier algorithms
\usepackage{algorithm}


%%%%%%%%%%%%%%%%%
%  Referencing  %
%%%%%%%%%%%%%%%%%

\usepackage{hyperref} 			% Links
\usepackage{cleveref} 			% References


%%%%%%%%%%%
%  Other  %
%%%%%%%%%%%

\usepackage{float}     			% For moving figures and other stuff around
\usepackage{titling}   			% For better titles 
\usepackage{csquotes}  			% Better quotes
\usepackage{scalerel} 			% For scaling symbols
\usepackage[margin, draft]{fixme}


%%%%%%%%%%%%%%
%  Commands  %
%%%%%%%%%%%%%%

% Basic shortcuts
\newcommand\N{\ensuremath{\mathbb{N}}}
\newcommand\R{\ensuremath{\mathbb{R}}}
\newcommand\Z{\ensuremath{\mathbb{Z}}}
\newcommand\Q{\ensuremath{\mathbb{Q}}}
\newcommand\C{\ensuremath{\mathbb{C}}}
\newcommand\E{\ensuremath{\mathbb{E}}}
\renewcommand\P{\ensuremath{\mathbb{P}}}
\renewcommand\O{\ensuremath{\emptyset}}


% Make implies and impliedby shorter
\let\implies\Rightarrow
\let\impliedby\Leftarrow
\let\iff\Leftrightarrow
\let\epsilon\varepsilon
\let\temp\phi
\let\phi\varphi
\let\varphi\temp

% Add \contra symbol to denote contradiction
\usepackage{stmaryrd} 			% for \lightning
\newcommand\contra{\scalebox{1.5}{$\lightning$}}

% Put x \to \infty below \lim
\let\svlim\lim\def\lim{\svlim\limits}

% Other commands
\DeclareMathOperator{\Span}{span}
\DeclareMathOperator{\given}{\; \vert \;}
\DeclareMathOperator*{\argmax}{arg\,max}
\DeclareMathOperator*{\argmin}{arg\,min}
\DeclareMathOperator{\ceq}{\coloneqq}
\DeclareMathOperator{\var}{Var}
\DeclareMathOperator{\e}{\mathrm{e}}
\DeclareMathOperator{\dom}{dom}
\DeclareMathOperator{\cf}{\varphi}
\DeclareMathOperator{\trace}{trace}


% make commands for abs and norm
\providecommand{\abs}[1]{\lvert#1\rvert}
\providecommand{\norm}[1]{\lVert#1\rVert}

%%%%%%%%%%%%%%%%%%
%  Environments  %
%%%%%%%%%%%%%%%%%%

\makeatother

% Box around theorems, etc.
% \usepackage{mdframed}
% \mdfsetup{skipabove=1em,skipbelow=0em}

% Options for exam theoremstyle (no numbers, english)
\theoremstyle{plain}
\newtheorem*{theorem}{Theorem}
\newtheorem*{proposition}{Proposition}
\newtheorem*{lemma}{Lemma}
\newtheorem*{corollary}{Corollary}

\theoremstyle{definition}
\newtheorem*{definition}{Definition}
\newtheorem*{algo}{Algorithm}
\newtheorem*{problem}{Problem}
\newtheorem*{example}{Example}
\newtheorem*{assumption}{Assumption}
\newtheorem*{claim}{Claim}

\theoremstyle{remark}
\newtheorem*{remark}{Remark}
\newtheorem*{Note}{Note}
\newtheorem*{obs}{\textbf{OBS}}

% Change proofname from "bevis" to "proof"
\renewcommand*{\proofname}{Proof}

% Make new command for solutions to exercises
\newenvironment*{solution}{\paragraph{Løsning:}}{\null\hfill$\diamond$}

% Add diamond at end of examples
\AtEndEnvironment{example}{\null\hfill$\diamond$}%


% Fix some spacing
% http://tex.stackexchange.com/questions/22119/how-can-i-change-the-spacing-before-theorems-with-amsthm
\makeatletter
\def\thm@space@setup{%
  \thm@preskip=\parskip \thm@postskip=0pt
}

% Exercises
% Use \exercise{num} to define exercise
% and \subexercise{num} to define subexercise
\newcommand{\exercise}[1]{%
    \def\@exercise{#1}%
    \subsection*{Opgave #1}
}

\newcommand{\subexercise}[1]{%
    \subsubsection*{Opgave \@exercise.#1}
}


% \lecture starts a new lecture 
%
% Usage:
% \lecture{1}{date}{title}
%
% This adds a section heading with the number / title of the lecture and a
% margin paragraph with the date.

% I use \dateparts here to hide the year. This way, I can easily parse
% the date of each lecture unambiguously while still having a human-friendly
% short format printed to the pdf.

\usepackage{xifthen}
\def\testdateparts#1{\dateparts#1\relax}
\def\dateparts#1 #2 #3 #4 #5\relax{
    \marginpar{\small\textsf{\mbox{#1 #2 #3 #5}}}
}

\def\@lecture{}%
\newcommand{\lecture}[3]{
    \ifthenelse{\isempty{#3}}{%
        \def\@lecture{Lecture #1}%
    }{%
        \def\@lecture{Lecture #1: #3}%
    }%
    \subsection*{\@lecture}
    \marginpar{\small\textsf{\mbox{#2}}}
}

\usepackage{xifthen}
\def\testdateparts#1{\dateparts#1\relax}
\def\dateparts#1 #2 #3 #4 #5\relax{
    \marginpar{\small\textsf{\mbox{#1 #2 #3 #5}}}
}

\def\@te{}%
\newcommand{\te}[3]{
    \ifthenelse{\isempty{#3}}{%
        \def\@te{TEs #1}%
    }{%
        \def\@te{TEs #1: #3}%
    }%
    \subsection*{\@te}
    \marginpar{\small\textsf{\mbox{#2}}}
}


% These are the fancy headers
%\usepackage{fancyhdr}
%\pagestyle{fancy}
%
%% LE: left even
%% RO: right odd
%% CE, CO: center even, center odd
%% My name for when I print my lecture notes to use for an open book exam.
%% \fancyhead[LE,RO]{Jens Trolle}
%
%\fancyhead[RO,L]{\@lecture} % Right odd,  Left even
%\fancyhead[R,LO]{}          % Right even, Left odd
%
%\fancyfoot[RO,L]{\thepage}  % Right odd,  Left even
%\fancyfoot[R,LO]{}          % Right even, Left odd
%\fancyfoot[C]{\leftmark}     % Center

\makeatother

% Todonotes and inline notes in fancy boxes
\usepackage{todonotes}
\usepackage{tcolorbox}

% Make boxes breakable
\tcbuselibrary{breakable}

% Rettelse is correction in Danish
% Usage:
% \begin{rettelse}
%     Lorem ipsum dolor sit amet, consetetur sadipscing elitr, sed diam nonumy eirmod
%     tempor invidunt ut labore et dolore magna aliquyam erat, sed diam voluptua. At
%     vero eos et accusam et justo duo dolores et ea rebum. Stet clita kasd gubergren,
%     no sea takimata sanctus est Lorem ipsum dolor sit amet.
% \end{rettelse}
\newenvironment{rettelse}{\begin{tcolorbox}[
    arc=0mm,
    colback=white,
    colframe=green!60!black,
    title=Rettelse,
    fonttitle=\sffamily,
    breakable
]}{\end{tcolorbox}}

% Note. Same as 'rettelse' but color of box is different
\newenvironment{note}[1]{\begin{tcolorbox}[
    arc=0mm,
    colback=white,
    colframe=white!60!black,
    title=#1,
    fonttitle=\sffamily,
    breakable
]}{\end{tcolorbox}}



% Fix some stuff
% %http://tex.stackexchange.com/questions/76273/multiple-pdfs-with-page-group-included-in-a-single-page-warning
\pdfsuppresswarningpagegroup=1

%%%%%%%%%%%%%%%%%%%%%%%%%%%%%%%%%%%
%  If TK set sum to other symbol  %
%%%%%%%%%%%%%%%%%%%%%%%%%%%%%%%%%%%

\let\temp\sum
\DeclareMathOperator*{\TKS}{\scalerel*{\rotatebox{0}{Z}}{\temp}}
\newboolean{TK}
\setboolean{TK}{false}
\ifthenelse{\boolean{TK}}{\let\sum\TKS}{\let\sum\temp}

% Name
\author{Jens Trolle}
