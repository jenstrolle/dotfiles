\documentclass[a4paper,oneside,article]{memoir}

\usepackage[utf8]{inputenc}  % Korrekt håndtering af æ, ø og å
\usepackage[T1]{fontenc}  % Korrekt håndtering af æ, ø og å
\usepackage{microtype}  % Typografisk magi! Giver bl.a. pænere orddeling
\usepackage{graphicx}  % Gør det muligt at indsætte billeder
\usepackage{amsmath}  % Giver adgang til uundværlige matematikting
\usepackage{amssymb,amsmath,amsfonts}
\usepackage{amsthm}
\usepackage[english]{babel}  % Danske betegnelser og orddeling
%\renewcommand{\danishhyphenmins}{22}  % Bedre dansk orddeling
\usepackage{listings}
\usepackage{xfrac} % Seje brøker
\usepackage{bm} % Mere fedt matematik
\usepackage{comment} % Bedre kommentarer
\usepackage[notextcomp]{kpfonts} % M&I symboler
\usepackage{tikz}  % Tegne grafer
\usepackage{float}
\usepackage{titling}
\usepackage{hyperref}
\usepackage{cleveref}
\usepackage{csquotes}
\usepackage[linesnumbered,lined,boxed,commentsnumbered]{algorithm2e}
\usepackage{algpseudocode}

% commands
\newcommand\given{\;\vert\;}
\DeclareMathOperator*{\argmax}{arg\,max}
\DeclareMathOperator*{\argmin}{arg\,min}

% Options for exam theoremstyle (no numbers, danish)
\theoremstyle{plain}
\newtheorem*{theorem}{Sætning}
\newtheorem*{proposition}{Proposition}
\newtheorem*{lemma}{Lemma}
\newtheorem*{korollar}{Korollar}

\theoremstyle{definition}
\newtheorem*{definition}{Definition}
\newtheorem*{alg}{Algoritme}
\newtheorem*{problem}{Problem}



